% Chapter 1


\begin{savequote}[10pc]
\sffamily
Smoke whirls\\
After the passage of a train.\\
Young foliage.
\qauthor{Shiki Masaoka (1867-1902)}
\end{savequote}
\chapter{\LaTeX{} info} % Write in your own chapter title

\label{Info}

\section{Examples of \LaTeX{} Typeset eBooks}

You can see (and download and read) examples of eBooks typeset by \LaTeX{} in a small library here:\\
\href{http://www.sunilpatel.co.uk/typesetting.html}{\texttt{http://www.sunilpatel.co.uk/typesetting.html}}

The books are available for you to download for free. They were originally plain text files but were quickly and easily converted into beautifully typeset and well presented eBooks through \LaTeX{} for you to enjoy.


\section{Learning \LaTeX{}}

\LaTeX{} is not a WYSIWYG (What You See is What You Get) program, unlike word processors such as Microsoft Word or Corel WordPerfect. Instead, a document written for \LaTeX{} is actually a simple, plain text file that contains \emph{no formatting}. You tell \LaTeX{} how you want the formatting in the finished document by writing in simple commands amongst the text, for example, if I want to use \emph{italic text for emphasis}, I write the `$\backslash$\texttt{emph}\{\}' command and put the text I want in italics in between the curly braces. This means that \LaTeX{} is a ``mark-up'' language, very much like HTML.

\subsection{A (not so short) Introduction to \LaTeX{}}

If you are new to \LaTeX{}, there is a very good eBook -- freely available online as a PDF file -- called, ``The Not So Short Introduction to \LaTeX{}''. The book's title is typically shortened to just ``lshort''. You can download the latest version (as it is occasionally updated) from here:\\
\href{http://www.ctan.org/tex-archive/info/lshort/english/lshort.pdf}{\texttt{http://www.ctan.org/tex-archive/info/lshort/english/lshort.pdf}}

It is also available in several other languages. Find yours from the list on this page:\\
\href{http://www.ctan.org/tex-archive/info/lshort/}{\texttt{http://www.ctan.org/tex-archive/info/lshort/}}

It is recommended to take a little time out to learn how to use \LaTeX{} by creating several, small `test' documents. Making the effort now means you're not stuck learning the system when what you \emph{really} need to be doing is writing your thesis.

\subsection{A Short Math Guide for \LaTeX{}}

If you are writing a technical or mathematical thesis, then you may want to read the document by the AMS (American Mathematical Society) called, ``A Short Math Guide for \LaTeX{}''. It can be found online here:\\
\href{http://www.ams.org/tex/amslatex.html}{\texttt{http://www.ams.org/tex/amslatex.html}}\\
under the ``Additional Documentation'' section towards the bottom of the page.

\subsection{Common \LaTeX{} Math Symbols}
There are a multitude of mathematical symbols available for \LaTeX{} and it would take a great effort to learn the commands for them all. The most common ones you are likely to use are shown on this page:\\
\href{http://www.sunilpatel.co.uk/latexsymbols.html}{\texttt{http://www.sunilpatel.co.uk/latexsymbols.html}}

You can use this page as a reference or crib sheet, the symbols are rendered as large, high quality images so you can quickly find the \LaTeX{} command for the symbol you need.


\section{Thesis Features and Conventions}\label{ThesisConventions}

To get the best out of this template, there are a few conventions that you may want to follow.

One of the most important (and most difficult) things to keep track of in such a long document as a thesis is consistency. Using certain conventions and ways of doing things (such as using the Todo list) makes the job easier. Of course, all of these are optional and you can adopt your own method.

\subsection{References}

The `\texttt{natbib}' package is used to format the bibliography and inserts references such as this one.\citep{Reference3} The options used in the `\texttt{Thesis.tex}' file mean that the references are listed in numerical order as they appear in the text. Multiple references are rearranged in numerical order\citep{Reference2, Reference1} and multiple, sequential references become reformatted to a reference range.\citep{Reference2, Reference1, Reference3} This is done automatically for you. To see how you use references, have a look at the `\texttt{Chapter1.tex}' source file.

References should come \emph{after} the punctuation mark if there is one (such as a comma or full stop). On the other hand, footnotes\footnote{Such as this footnote, here down at the bottom of the page.} come \emph{before} the punctuation mark. You can swap these around but the most important thing is to keep the convention consistent throughout the thesis. Footnotes themselves should be full, descriptive sentences (beginning with a capital letter and ending with a full stop).

To see how \LaTeX{} typesets the bibliography, have a look at the very end of this document (or just click on the reference number links).

\subsection{Figures}

There will hopefully be many figures in your thesis (that should be placed in the `Figures' folder). The way to insert figures into your thesis is to use a code template like this:
\begin{verbatim}
\begin{figure}[htbp]
  \centering
    \includegraphics{./Figures/Electron.pdf}
    \rule{35em}{0.5pt}
  \caption[An Electron]{An electron (artist's impression).}
  \label{fig:Electron}
\end{figure}
\end{verbatim}
Also look in the source file. Putting this code into the source file produces the picture of the electron that you can see in the figure below.

\begin{figure}[htbp]
	\centering
		\includegraphics{./Figures/Electron.pdf}
		\rule{35em}{0.5pt}
	\caption[An Electron]{An electron (artist's impression).}
	\label{fig:Electron}
\end{figure}

Sometimes figures don't always appear where you write them in the source. The placement depends on how much space there is on the page for the figure. Sometimes there is not enough room to fit a figure directly where it should go (in relation to the text) and so \LaTeX{} puts it at the top of the next page. Positioning figures is the job of \LaTeX{} and so you should only worry about making them look good!

Figures usually should have labels just in case you need to refer to them (such as in figure \ref{fig:Electron}). The `$\backslash$\texttt{caption}' command contains two parts, the first part, inside the square brackets is the title that will appear in the `List of Figures', and so should be short. The second part in the curly brackets should contain the longer and more descriptive caption text.

The `$\backslash$\texttt{rule}' command is optional and simply puts an aesthetic horizontal line below the image. If you do this for one image, do it for all of them.

The \LaTeX{} Thesis Template is able to use figures that are either in the PDF or JPEG file format. It is recommended that you read this short guide on how to get the best out of figures in \LaTeX{}, available here:\\
\href{http://www.sunilpatel.co.uk/texhelp5.html}{\texttt{http://www.sunilpatel.co.uk/texhelp5.html}}

Though it is geared more towards users of Mac and OS X systems, much of the advice applies to creating and using figures in general. It also explains why the PDF file format is preferred in figures over JPEG.

\subsection{Typesetting mathematics}

If your thesis is going to contain heavy mathematical content, be sure that \LaTeX{} will make it look beautiful, even though it won't be able to solve the equations for you.

The ``Not So Short Introduction to \LaTeX{}'' (available \href{http://www.ctan.org/tex-archive/info/lshort/english/lshort.pdf}{here}) should tell you everything you need to know for most cases of typesetting mathematics. If you need more information, a much more thorough mathematical guide is available from the AMS called, ``A Short Math Guide to \LaTeX{}'' and can be downloaded from:\\
\href{ftp://ftp.ams.org/pub/tex/doc/amsmath/short-math-guide.pdf}{\texttt{ftp://ftp.ams.org/pub/tex/doc/amsmath/short-math-guide.pdf}}

There are many different \LaTeX{} symbols to remember, luckily you can find the most common symbols \href{http://www.sunilpatel.co.uk/latexsymbols.html}{here}. You can use the web page as a quick reference or crib sheet and because the symbols are grouped and rendered as high quality images (each with a downloadable PDF), finding the symbol you need is quick and easy.

You can write an equation, which is automatically given an equation number by \LaTeX{} like this:
\begin{verbatim}
\begin{equation}
E = mc^{2}
  \label{eqn:Einstein}
\end{equation}
\end{verbatim}

This will produce Einstein's famous energy-matter equivalence equation:
\begin{equation}
E = mc^{2}
\label{eqn:Einstein}
\end{equation}

All equations you write (which are not in the middle of paragraph text) are automatically given equation numbers by \LaTeX{}. If you don't want a particular equation numbered, just put the command, `$\backslash$\texttt{nonumber}' immediately after the equation.


\section{Sectioning and Subsectioning}

You should break your thesis up into nice, bite-sized sections and subsections. \LaTeX{} automatically builds a table of Contents by looking at all the `$\backslash$\texttt{chapter}$\{\}$', `$\backslash$\texttt{section}$\{\}$' and `$\backslash$\texttt{subsection}$\{\}$' commands you write in the source.

The table of Contents should only list the sections to three (3) levels. A `$\backslash$\texttt{chapter}$\{\}$' is level one (1). A `$\backslash$\texttt{section}$\{\}$' is level two (2) and so a `$\backslash$\texttt{subsection}$\{\}$' is level three (3). In your thesis it is likely that you will even use a `$\backslash$\texttt{subsubsection}$\{\}$', which is level four (4). Adding all these will create an unnecessarily cluttered table of Contents and so you should use the `$\backslash$\texttt{subsubsection$^{*}\{\}$}' command instead (note the asterisk). The asterisk ($^{*}$) tells \LaTeX{} to omit listing the subsubsection in the Contents, keeping it clean and tidy.


\section{In Closing}

You have reached the end of this mini-guide. You can now rename or overwrite this pdf file and begin writing your own `\texttt{Chapter1.tex}' and the rest of your thesis. The easy work of setting up the structure and framework has been taken care of for you. It's now your job to fill it out! If you use this Thesis template and this mini-guide helps you, please let me know\footnote{Email: \href{mailto:web@sunilpatel.co.uk}{web@sunilpatel.co.uk} all comments and suggestions, questions and errata are welcome.}.

Good luck and have lots of fun!

\begin{flushright}
Sunil Patel: \href{http://www.sunilpatel.co.uk}{www.sunilpatel.co.uk}
\end{flushright}
