\part{The Clients}
%% A Web Service client is the consumer of a Web Service. This part of the
%% dissertation describes the design and implementation of clients that consume the
%% Web Services described in the last part.

%% There are two clients meant for user consumption: the main front-end ()of the
%% application manifested in an Ajax client and a supplementary front-end manifested
%% in a Django client. The former is apt for standard computer whereas the latter is
%% apt for mobiles.
\begin{savequote}[10pc]
\sffamily
``It is a capital mistake to theorize before one has data.''
\qauthor{Sherlock Holmes, Sir Arthur Conan Doyle (1859 -- 1930)}
\end{savequote}
\chapter{Weather data}\label{chap:forecasts}

\section{Getting weather data}\label{sec:getting_weather_data}
The resources incorporate weather data -- forecasts and observations -- from four
external resources: DCA, DMI, WB, and NOAA. The weather data must be up to data
when the user requests the information. None of the sources provide JSON(P) 
Web Services therefore we must either use the GAE as a proxy or save the data
locally and periodically update the content. 7 April 2009 Google announced
support for scheduling cron
jobs\footnote{\url{http://code.google.com/p/googleappengine/issues/detail?id=6}},
giving three options for keeping the data up to date: either update the data
\begin{enumerate}
  \item by an internal cron job at the GAE;
  \item on-demand upon user HTTP requests; or,
  \item by an external cron job.
\end{enumerate}
%%
\textit{Internal cron jobs:} The natural choice would be an internal cron job at
the GAE, making the GAE responsible for consuming weather data from the external
resources at regular intervals. The cron jobs at Google are limited to the same
constraints as usual HTTP requests, e.g., the request duration limit. In
addition, the number of jobs are limited to 20, and at minimum there must be one
minute between individual cron jobs \citep{Google:cron}. With hundreds of weather
stations a naive cron job times out, and the changes done in the transaction are
rolled back. Since it is possible to execute cron jobs every minute, a workaround
is to limit the number of processed weather stations. Every data retrieval lasts
about one second which means that about 20 resources can be processed per minute
under the request duration limit. Weather stations are updated ca. every 20
minutes, which yields only 400 stations that can be updated on a continuously
basis per cron jobs, and which makes this approach infeasible.  \\\\
\textit{On-demand:} By on-demand we mean weather data is updated just-in-time
when the user needs the weather data. Because of the request duration limit the
approach would include requests from JavaScript at the client side for every
resource of interest, and the controller would update the GAE version of the
weather data if it is stale. The approach, however, includes a significant
latency. Getting weather data must be a low latency operation making the approach
infeasible.\footnote{GAE is an emerging technology. A promising GAE offering
(announced 18 June 2009) is the Task Queue. The Task Queue process tasks
independently of the originating request. The only problem is getting the updated
information back to the user, and not just to the next user accessing the
page. Such a requirement is typically realized with Comet (or Ajax Push),
however, this is not supported on GAE. If (or when) it is supported the Task
Queue together with Comet is a solution for the update problem.}  \\\\
\textit{External cron jobs:} The last option is letting external cron jobs pull
weather data from all the weather resource and push the data to the GAE. The
external updaters are tightly coupled to the GAE in a point-to-point solution
with much redundant data transfer, however, it is the best of the worst at the
moment.

\section{Weather Observations}\label{chap:observations}
Unfortunately we have no Swiss Army Knife for the problem of pulling information
from the different sources. The resources are diverse and in different formats;
therefore, each case is handled on an individual basis. 

\subsection{Scraping DMI and DCA}
The only option to get weather data from DCA and DMI is by scraping their Web
pages for content. Web scraping is the process of programmatically retrieving
content from a Web page meant for human consumption. The data at DCA is
inaccessible as a Web Service, but DCA has agreed to let their pages be accessed
by our Web Scraper in order to retrieve the data.

DMI prohibits incorporation of their data in other sites. The author has several
times asked DMI for a collaboration, however, un-successful. We might be crossing
ethical boundaries, but, we have decided to Web scrape their site, just like
Google does.

Content in the case of DCA is located at sub-pages on a station and data type
basis (speed, direction, and temperature are located at different pages). Content
at DMI is located in sub-pages on a weather station basis. 

All resources are invalid HTML and will cause an XML parser to fail. This is a
case where Beautiful
Soup\footnote{\url{http://www.crummy.com/software/BeautifulSoup/}}, a Python
module to Web scrape Web pages, has a great force; to quote their Web page:
%
\begin{quote}
\footnotesize\itshape
You didn't write that awful page. You're just trying to get some data out of
it. Right now, you don't really care what HTML is supposed to look like.\\
\\
Neither does this parser.
\end{quote}
%
Scraping the latest speed, direction, temp, and time from DMI Aarhus, the code in
Listing~\ref{lst:dmiscraper} is all that is needed (after downloading the
BeautifulSoup module).

\begin{lstlisting}[caption=DMI Web scraper, label=lst:dmiscraper]
months = ['', 'januar', 'februar', 'marts', 'april', 'maj', 'juni', 'juli', 'august', 'septempber', 'oktober', 'november', 'december']

def _scrape(ws):
    page = urlopen(ws['uri_scrape']).read()
    soup = BeautifulSoup(page)

    # all text from page
    text = soup.findAll(text=True)

    # time
    contents = text[6].split()
    day = int(contents[2].strip('.'))
    month = int(months.index(contents[3])) # januar -> 1
    year = int(contents[4])
    hour = int(contents[5].split(':')[0])
    minute = int(contents[5].split(':')[1])
    dt = datetime(year, month, day, hour, minute)
    iso_time_utc = convert_local_time_to_utc(dt.timetuple())

    # wind
    contents = text[7].split()
    direction = float(contents[1])
    speed = float(contents[3])

    # temp
    contents = text[8].split()
    temp = float(contents[1])

    return {'time':iso_time_utc, 'direction':direction, 'temp':temp, 'speed':speed}

def convert_local_time_to_utc(time_tuple):
    secs = time.mktime(time_tuple)
    utc = time.gmtime(secs)
    iso = time.strftime("%Y-%m-%dT%H:%M:%S", utc)
    return iso

if __name__ == '__main__':
    print _scrape({'uri_scrape':'http://servlet.dmi.dk/bv/servlet/bv?stat=6074'})
\end{lstlisting}
%
In the example we use the Python library \verb|urllib2| to fetch the page. This
is given as an argument to BeatifulSoup when it is instantiated. The
\verb|BeautifulSoup| class has many methods that assist in extracting relevant
data from the page; we use the \verb|findAll| method to find all elements, and
extract content from that list.

Scraping DCA's site is similar except that it needs tailoring for the specific
layout of that site. 

The Web scraper is fragile to modifications in the user interface, e.g.,
modification to the placement of the different data, but we have no other option
to retrieve the data.

\subsection{Scraping WB}\label{client:weatherbug}
In contrast to the ill-formed data from DCA the data from WB is in XML formats,
and therefore more apt for computer consumption. Listing~\ref{lst:weatherbug_rss}
shows an example of a WB XML document. The documents are subject to a single
operation: extraction of node content and node attribute values.

From an abstract viewpoint, there are two types of XML parsers: tree-based
parsers and event-based parsers. Tree-based parsers read in the whole XML
document at once and create an internal data structure, while event-based parsers
consider XML documents as a stream of events serially generating events while
parsing the document. In the previous section, we used a tree-based parser,
BeautifulSoup. The advantages of event-based frameworks over tree-based are that
they are fast and efficient since they avoid building a tree data structure;
the advantage, however, is at the expense of simplicity in the programming model.

XML parsing of WB data merely has the purpose of extracting content of specific
nodes: the application does not call for any transformations, and not for
processing dependent on former seen nodes, and not for fixing invalid XML;
therefore, we choose the event-based parser, SAX.\footnote{The Python standard
library contains both types of XML parsers: tree-based, manifested in the
Document Object Model\footnote{\url{http://docs.python.org/library/xml.dom.html}}
(DOM), and event-based, manifested in the Simple API for
XML\footnote{\url{http://docs.python.org/library/xml.sax.html}} (SAX).}

\begin{lstlisting}[caption=WeatherBug Weather RSS feed for weather station Tirstrup Airport,label=lst:weatherbug_rss]
(...)
  <aws:station requestedID="EKAH" id="EKAH" 
    name="Tirstrup Airport" 
    city="Tirstrup"  state="Denmark" 
    citycode="60090" country="Denmark" 
    latitude="56.2999992370605" 
    longitude="10.6166667938232" />
  <aws:current-condition 
    icon="http://deskwx.weatherbug.com/images/Forecast/icons/cond024.gif">
    Mostly Cloudy
  </aws:current-condition>
  <aws:temp units="&amp;deg;C">9.0</aws:temp>
  <aws:rain-today units="mm">0.00</aws:rain-today>
  <aws:wind-speed units="km/h">22</aws:wind-speed>
  <aws:wind-direction>ESE</aws:wind-direction>
  <aws:gust-speed units="km/h">21</aws:gust-speed>
  <aws:gust-direction>ESE</aws:gust-direction>
</aws:weather>
(...)
<pubDate>Wed, 15 Apr 2009 13:50:00 GMT</pubDate>
(...)
\end{lstlisting}

%% http://api.wxbug.net/getStationsXML.aspx?ACode=A5580954882&lat=56.15&long=10.21

Parsing with Python SAX is done by inheriting from the class
\verb|ContentHandler| and overwriting relevant methods. The methods correspond to
events generated when SAX parses the document.

A generic content handler that extract content must overwrite three methods in
the \verb|ContentHandler|: \verb|startElement|, \verb|endElement|, and
\verb|characters|. During parsing the \verb|startElement| method is called for
every XML start tag, the \verb|endElement| method is called for every XML end
tag, and the \verb|characters| method is called for each chunk of character
data.\footnote{For more info see \url{http://docs.python.org/library/xml.sax.handler.html}}

Listing~\ref{lst:sax_xmlparser} shows the code for the XML parser. The parser is
instantiated with a list of the type of XML elements from which to retrieve the
content. The parser keeps track of when it is inside relevant nodes with the
\verb|in_element| variable. The relevant node content is put in a dictionary of
XML-element keys and node content values.

% XML SAX Parser 
\lstinputlisting[caption=SAX XML parser,label=lst:sax_xmlparser]{../implementation/wlw/wlw/helpers/xmlparser.py}

Extracting the content of the RSS feed is now just a matter of picking the
elements to retrieve the node content from, and parsing the XML string with the
content handler, the code is shown in Listing~\ref{lst:weatherbug_scrape}.

\begin{lstlisting}[caption=Extracting XML content from WB station RSS feed,label=lst:weatherbug_scrape]
XML_ELEMENTS = ['aws:temp', 'aws:wind-speed', 'aws:wind-direction','aws:gust-speed', 'aws:gust-direction', 'pubDate']

def _scrape(ws):
    response = urllib2.urlopen(ws['uri_scrape']).read()
    values = {}
    parseString(response, XMLParser(values, XML_ELEMENTS))

    time = convert_local_time_to_utc(parse_wb_time(values['pubDate']))
    temp = parse_float(values.get('aws:temp'))
    gust = parse_float(values.get('aws:gust-speed'))
    direction = values.get('aws:wind-direction')
    speed = parse_float(values.get('aws:wind-speed'))

    if speed:
        speed = speed / 3.6 # from km/h to m/s

    if gust:
        gust = gust / 3.6

    if direction:
        direction = convert_direction(direction)

    return {'time':time, 'direction':direction, 'temp':temp, 'speed':speed, 'gust':gust}

    (...)

if __name__ == '__main__':
    print _scrape({'uri_scrape':'http://api.wxbug.net/getLiveCompactWeatherRSS.aspx?ACode=A5580954882&stationid=ETGG&unittype=1'})
\end{lstlisting}

\subsection{Integrating with the Web Service}
Web scraping the three types of weather stations is a similar process: 
\begin{enumerate}
  \item fetch all weather stations of a certain type from the Web Service; and
  for all weather stations,
  \item scrape it and post the new data to the Web Service.
\end{enumerate}
The only difference between the weather station types are in the concrete
scraping code. Therefore, we apply the template method pattern \citep{Gamma:1995}
and create a general scraper skeleton, deferring the concrete scraping code to
its subclasses.

\verb|update_weather_data| is the template method, shown in
Listing~\ref{lst:scraper}, that implements the process described above.
\verb|_scrape| is the deferred method, it uses the \verb|scrape_uri| from the
fetched representation and returns the scraped data as a simple Python object
compatible with the observation representation format
(see~\vref{lst:obs_res}). \verb|_update_server| converts the object to JSON, sets
the content type (used by the controller in the Web Service), and post the
representation to the server. \verb|_scrape| continues recursively if the
paginator contains more pages. The individual \verb|_scrape| methods were
presented in the last section and fit directly into the skeleton when they are
put in a class that inherits from \verb|Scraper|.

\begin{lstlisting}[caption=Common Web scraper,label=lst:scraper]
URI_BASE = 'http://www.welovewind.com'
URI_WEATHER_STATIONS = 'http://www.welovewind.com/api/weather_stations/'
class Scraper:

    def __init__(self, type):
        '''Craete instance of scraper
        Args:
          type: weather station type ('dmi'|'wb'|'dca')
        '''
        self.type = type

    def update(self):
        self.update_weather_data(URI_WEATHER_STATIONS + '?type=%s' % self.type)

    def update_weather_data(self, uri):
        '''Template method'''
        wss = self._fetch_weather_stations(uri)
        for ws in wss['items']:
            data = self._scrape(ws)

            # update server
            uri = '%s%s' % (URI_BASE, ws['uri_observations'])
            self._update_server(uri, data)
        if wss['next']:
            self.update_weather_data('%s%s' % (URI_BASE,wss['next']))

    def _fetch_weather_stations(self, uri):
        response = urllib2.urlopen(uri)
        return simplejson.load(response)


    def _scrape(self, ws):
        raise Exception('Abstract Method')

    def _update_server(self, uri, data):
        try:
            json = simplejson.dumps(data)
            # putting in the values will cause the request to be a POST
            request = urllib2.Request(uri, json, {'Content-Type': 'application/json'})
            response = urllib2.urlopen(request).read()
        except urllib2.HTTPError, e:
            logging.error('trying to post data: %s' % data)


\end{lstlisting}

We use \verb|cron| as the job scheduler. A table that runs the scraper each 20 minutes
looks like the following:
\begin{verbatim}
0,20,40 * * * * /scraper.py
\end{verbatim}
After installing the table with \verb|crontab -e| in a Unix like environment
weather observations are continuously updated for all the weather stations in the
Web Service.

\section{Forecasts}
We stated earlier that there was no Swiss Army Knife for doing Web Scraping, this
is even more true for retrieving the forecasts. 

\subsection{Retrieving forecasts}
The forecasts are quite large about 45 MB for a single forecast. This means a
total download of 45 MB * (48 hours/ 3 hour interval) = 720 MB every 6 hours! 3
fields (temperature, u wind component, v wind component) out of 250 are
interesting for the application. It is possible to reduce the download to
interesting data only. This is done by first downloading a small -- around 26KB
-- inventory of the forecast; it specifies the content of the forecast and at
which byte ranges the different content is located. This makes it possible to
only fetch the parts we are interested in. A perl script that does this is
available.\footnote{\url{http://www.cpc.noaa.gov/products/wesley/fast_downloading_grib.html}}

A small shell script using perl scripts that downloads a subset of fields,
now with a size under 1 Mb, looks as follows:
\begin{verbatim}
#!/bin/sh

url=http://nomad3.ncep.noaa.gov/pub/gfs/rotating-0.5/gfs.t00z.pgrb2f00
get_inv.pl "${url}.inv" | \
egrep "(:UGRD:10 m above|:VGRD:10 m above|:TMP:surface)" | \
get_grib.pl "${url}" pgb.grb  
\end{verbatim}

The NOAA server contains forecasts from four calculation runs which are
circularly updated as the forecasts calculations are finished. The application
keeps a local version of all forecasts files, and also updates them
circularly. Local forecasts are updated when they are stale by comparing the last
modified time -- found with a HTTP HEAD request -- of the remote files with the
local files.\footnote{We note that we have changed the perl scripts in order to
preserve the external time when downloading the forecasts.} In the application we
check the last file in each run (since it is produced by NOAA after the other
files in the run) and update all forecasts in the run if the last is stale. The
updating logic checks last modification times twice every hour by a cronjob.

\subsection{Extracting data from GRIB2 files}
\verb|wgrib2|\footnote{wgrib is found at:
\url{http://www.cpc.noaa.gov/products/wesley/wgrib2/}} is a program that
retrieves data from GRIB2 files. \verb|wgrib2| supports extracting records
out of a GRIB2 file based on latitude and longitude. We can type in the following
command to get all data fields about Skagen from a GRIB2 file.

\begin{verbatim}
$ wgrib2 forecasts/gfs.t00z.pgrb2f00 -s -lon 10.59 57.7
1:115:d=2009062700:TMP:2 m above ground:anl::lon=10.500000,
    lat=57.500000,val=289.71
2.1:180652:d=2009062700:UGRD:10 m above ground:anl::lon=10.500000,
    lat=57.500000,val=-4.81
2.2:180652:d=2009062700:VGRD:10 m above ground:anl::lon=10.500000,
    lat=57.500000,val=-0.76
\end{verbatim}

The \verb|-lon| option gets values from a grid point closest to a specified
point. The output shows the closest point, in this case it is (57.5,10.5),
together with the value of the records.

The values in the GRIB2 files are in a meteorological format not applicable for
the users of the Weather Web Service.

\subsubsection*{Temperature conversion}
The temperature is reported in Kelvin; conversion: 
\begin{math}
[�C] = [K] - 273.15
\end{math}

\subsubsection*{Wind velocity conversion}
The wind velocity forecast is a vector separated in two components \verb|u| and
\verb|v|. \verb|u| is the zonal -- East-West -- component; if positive the wind
is blowing to the East. \verb|v| is the meridional -- North-South -- component;
if positive the wind is blowing to the North.

Instead of the vectors the Web Service reports the wind speed and the
meteorological wind direction; which is a value in the interval [0;360[ with
respect to North (North=0), where the wind is coming from. The components are
converted in Python, shown in Listing~\ref{lst:conversions}.

\begin{lstlisting}[caption=Conversions in Python,label=lst:conversions]
import math
def toSpeed(u_component, v_component):
    """Returns the norm of the vectors, i.d., the speed.
    """
    return math.hypot(u_component,v_component)
    
def toDirection(u_component, v_component):
    degrees = math.degrees(math.atan2(-u_component, -v_component))
    if(degrees < 0):
        degrees += 360
    return degrees
\end{lstlisting}
%
\verb|u| and \verb|v| are negated since meteorological wind direction is where the
wind is coming from instead of blowing to. \verb|atan2| returns an angle in the
interval $[-\pi;\pi]$. To map this into $[0�;360�]$ we must add $360�$ to
negative angles.

\subsection{Integrating with the Web Service}
A central function for the updating process is retrieving all future forecasts
from files in the latest run. The function finds the latest run, comparing the
last forecast in each run. All file names in a run is available in a sorted list;
the time delta from the calculation time to the current utc time is used to slice
away stale forecasts from the list. \verb|get_wind_forecast| uses \verb|wgrib2|
on forecast files and returns weather data found with regular expression on the
output from \verb|wgrib2|.

\begin{lstlisting}
URI_BASE = 'http://www.welovewind.com'
URI_POINTS = 'http://www.welovewind.com/api/forecast_points/'

def update_server(uri):
    """PUT forecasts updates to server if the data at the server is stale.

    """
    forecast_points = _fetch_forecasts(uri)
    for p in forecast_points['items']:
        if not is_local_fresher(p):
            continue
        fp = forecasts.ForecastPoints(lat=p['lat'], lon=p['lon'], maximum=16)
        if p['lat'] != fp.lat or p['lon'] != fp.lon:
            raise Exception('lat/lon inconsistency')
        uri = '%s%s?_method=PUT' % (URI_BASE, p['uri'])
        representation = fp.to_json()
        request = urllib2.Request(uri, representation, {'Content-Type': 'application/json'})
        response = urllib2.urlopen(request).read()
    if forecast_points['next']:
        update_server('%s%s' % (URI_BASE, forecast_points['next']))

def _fetch_forecasts(uri):
    response = urllib2.urlopen('%s' % uri).read()
    return simplejson.loads(response)
\end{lstlisting}

\section{Summary}
In this chapter, we presented and used different types of Web scraping techniques
to fetch data from external resources not meant for computer consumption. In
addition, we presented logic to load data into the application on a continuous
basis.
