%% I can't change the direction of the wind, but I can adjust my sails to always reach my destination. 
%% Jimmy Dean 

% Chapter Conclusion
\part{Conclusion \& Perspectives}

%% \begin{savequote}[10pc]
%%  \sffamily
%% ``I can't change the direction of the wind, but I can adjust my sails to always reach my destination.''
%% \qauthor{James Dean (1931 -- 1955)}
%% \end{savequote}

%\chapter{Conclusion \& Perspectives} % Write in your own chapter title
\label{Conclusion}
Through three parts in this dissertation, The Foundations, The Web Services, and
The Clients, we have reached the destination, a ``Web-Based Weather Service for
Wind Sports''. The integration of these three parts is a mashup that assists the
practitioners of wind sports.

Our dissertation gives an insight into the course we took to 1) extract data from
external resources (independent of format) and, to 2) create a geographical
information system mashup based on that data. The insight includes:
\begin{itemize}
  \item The theoretical foundations for creating mashups
(Chapter~\ref{chap:web}). The foundations included a presentation of the
architectural style REST, the architecture ROA upon which we based our Web
Services, and an infrastructure to run the mashups: the GAE.
  \item Practices for designing the resources in a Web Services
(Chapter~\ref{chap:resources}) and afterwards practices for implementation of
mashups; in our application, manifested in two Web Services: the We Love Wind Web
Service (Chapter~\ref{chap:gae_ws}) and the DAIMI Forecast Web Service
(Chapter~\ref{chap:ws_daimi}). 
  \item At last the insight includes the theoretical foundations for creating an
  Ajax client (Chapter~\ref{chap:ws_clients}) and a mobile client. In addition,
  concrete practices for implementing such clients (Chapter~\ref{chap:client} and
  Chapter~\ref{chap:mobile})
\end{itemize}

Our thesis was only possible because we had access to public weather information
from the US that use the `open access' model \citep{noaa:datareport} for public
sector information. The European model, known as the `cost recovery' model,
restricts access to public data. During the last half year there have been
European initiatives that argue in favor for open access to public
data. \url{digitaliser.dk}, created by the Danish government, is one of the
front-runners, quoting their Web site:
\begin{quote}
Digitalis�r.dk aims to stimulate development and adoption of digital content and
business models by utilising Web 2.0 technologies and public data and digital
resources. 
\end{quote}
\verb|digitaliser.dk| aspires to create common grounds of how to get access to
public sector information. Times are thus changing in favor of mashups.

%% \begin{quotation}
%% \begin{flushright}
%% \textit{Come senators, congressmen\\
%% Please heed the call\\
%% Don't stand in the doorway\\
%% Don't block up the hall\\
%% For he that gets hurt\\
%% Will be he who has stalled\\
%% There's a battle outside\\
%% And it is ragin'.\\
%% It'll soon shake your windows\\
%% And rattle your walls\\
%% For the times they are a-changin'.\\}
%% Bob Dylan (1941--)
%% \end{flushright}
%% \end{quotation}

\part{Appendix \& Glossary}