\chapter{The DAIMI Forecast Web Service}\label{chap:ws_daimi}
In the last chapter we described the We Love Wind Web Service. That service
included external calls to another Web Service: The DAIMI Forecast Web Service, found
at
\begin{itemize}
  \item \url{http://daimi.welovewind.com/}
\end{itemize}
There are important requirements in the application that demands an additional
forecast Web Service:
\begin{itemize}
  \item it should be possible to get forecasts for arbitrary locations; and,
  \item after creating new spots, the forecasts of the nearest forecast point
  should be in place immediately. 
\end{itemize}
If the logic to get forecasts were placed on GAE, the requirements are
trivial. The application is based on binary forecasts from NOAA. The decoding
software is incompatible with GAE; the decoding of the binary forecast files
demands that the decoding logic is placed elsewhere. It is infeasible to
continuously update all forecast points -- ca. 260000 points -- therefore, the
forecasts of points are only updated when relevant for surf spots, and that
demands an extra service for forecasts. In essence, we are wrapping a modern Web
Service around the binary forecasts files from NOAA.

In this chapter we describe the implementation of a RESTful weather service with
Python/CGI.\footnote{The Web Service is located at the department of computer
science, Aarhus University; that left the author with two options for generating
the dynamic output: PHP or CGI. To keep things consistent we implement the
forecast service in a Python CGI script.} The focus of this chapter is on the
implementation of the Web Service; we postpone the description of retrieving and
converting weather data to the next part.

The resource of the service is already defined: it is the forecast points
resources described in Section~\vref{sec:res:forecast_points}. Therefore, we jump
directly to the implementation; again described in terms of the MVC style.

\section{Model}
The forecast weather service contains a single model: the forecast point
model. The purpose of the model is to encapsulate the access to the weather
forecasts located in files. 

\subsection*{Structure}
The model contains three properties: the latitude and longitude of the point, and
the forecasts of the point.

\subsection*{Logic}
The forecast point model is able to return a representation of itself in JSON
format. The main method in the model is a class method that retrieves the current
and all future forecast from the gfs files; since decoding forecasts from the
forecast files takes about 0.5 seconds the method takes a maximum argument that
reduces the number of returned forecasts.

\section{Controller}
 The controller is implemented as a Python/CGI script. CGI is a standard that
connects a server with any programming language. Responses are crafted by
printing the headers of the HTTP response which must include a
\verb|content-type| header, the header is ended with an empty line and after that
the program prints the body of the response.

Listing~\ref{lst:cgi_forecast_point_controller} shows the
controller. \verb|handle_request| is the entry method; it starts by extracting
the parameters of the request. The values of those parameters are later given to
an external shell program as arguments; therefore, the parameters are sanitized
and parsed as floats avoiding security issues. If any error occur during this
process the response signals the error with a HTTP 400 Bad Request. Secondly, the
extracted parameters are given to the model which populates itself with forecast
data according to the arguments. Finally, if caching does not apply the forecast
point is printed in JSON format. 

\begin{lstlisting}[caption=Forecast point CGI controller, label=lst:cgi_forecast_point_controller]
def handle_request():
    try:
        params = extract_params()
        if not params:
            print_intro()
            return
        forecasts = ForecastPoints(**params)
        if etag_match(forecasts):
            not_modified_304()
            return
        print_forecasts(forecasts)
    except (ValueError, IndexError):
        print_intro(400)

def extract_params():
    input = os.environ.get('REDIRECT_QUERY_STRING')
    lat, lon, maximum = None, None, 3 # maximum 3 if not specified
    if input:
        input = input.strip('/').split(',')
        lat = float(input[0])
        lon = float(input[1])
        if lat < -90.0 or lat > 90.0 or lon < -180.0 or lon > 180.0:
            raise ValueError
        match = re.search(r'max=(\d+)', os.environ.get('REQUEST_URI'))
        if match:
            maximum = int(match.group(1))
        return {'lat':lat, 'lon':lon, 'maximum':maximum}
    else:
        return None

def print_intro(status=None):
    print 'Content-type: text/plain'
    if status:
        print 'Status: %s' % status
    print
    print 'Welcome to the DAIMI Weather Forecasts Service'
    print 'forecasts are accessed by ..forecasts/{lat},{lon}'
    print 'the number of returned forecast is specified by setting the max query parameter: ../?max=10'

def print_forecasts(forecasts):
    print 'Content-type: text/plain'
    print 'ETag: %s' % forecasts.last_modified()
    print
    print forecasts.to_json()

def not_modified_304():
    print 'Status: 304 Not Modified'
    print

def etag_match(forecasts):
    etag = os.environ.get('HTTP_IF_NONE_MATCH')
    if etag:
        if forecasts.last_modified() == etag:
            return True
    return False

handle_request()  
\end{lstlisting}

\section{Summary}
In this chapter, we described the implementation of a forecast Web Service. We
designed two Web Services since NOAA forecasts was incompatible with GAE. At the
time of implementation the GAE lacked support for cron jobs, offline processing,
and Java. All the necessary components are now in place\footnote{Offline
processing of tasks is still experimental on GAE.} making a future solution that
is more clean and effective possible. Java is needed since there are limited
options to decode GRIB2 files: the unidata Java decoder
tools\footnote{\url{http://www.unidata.ucar.edu/software/decoders/}},
\verb|wgrib2|, and
\verb|pygrib2|\footnote{\url{http://pygrib2.googlecode.com/}}. The latter two are
incompatible with GAE since they are based on \verb|C| modules.

This ends the description of the Web Services (or the server-side) of the We Love
Wind mashup. In the following chapters we turn to the clients of those Web
Services.
